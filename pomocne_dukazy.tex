\documentclass{article}
\usepackage{amsmath}
\usepackage{amssymb}
\begin{document}
z předpokladu, že dva body každý leží na přímce, které se neprotínají chceme
dokázat, že axiom o přímkách impluje, že nemůžeme žádnou přímku prodloužit tak
aby spojila tyto dva body. 

Kdybychom totiž jakokoliv prodloužili vytvořili bychom tak druhý průsečík mezi
dvěma přímkami. Protože z a1 plyne, že ty dvě přímky jedna na které ležel bod do
kterého jsme prodloužili. a ta druhá kterou jsme prodloužili. už mají jeden
průsečík. My jsme přidali další.



Předpoklad je první axiom:

\[
 \forall p,q \in P: \exists! c \in B:p \cap q = c
\]


\[
 \exists a,b \in B:\nexists p \in  P: a \subseteq p \wedge b \subseteq p
\]
Další předpoklad jsou dva body, které neleží na jedné přímce, ale zároveň na
nějaké leží.
\[
 \exists p_a,q_b \in P: \exists a,b \in B: a \subseteq p_a \wedge b \subseteq
 q_b
\]
\[
 p_a \neq q_b \wedge a \neq b
\]
Přidejme tedy $b$ do $p_a$. Tím jsme porušili axiom 1 protože už předtím museli
mít přímky $p_a$ a $q_b$ právě jeden průsečík. Teď jsme, ale vytvořili další.
Jediné řešení této situace je, že by se body $c$ a $b$ rovnali. To by, ale
znamenalo, že $b$ a $a$ leželi na jedné přímce což jde proti jednomu z
předpokladů. 


\end{document}
