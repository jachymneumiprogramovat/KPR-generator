\documentclass{article}


\begin{document}
z předpokladu, že dva body každý leží na přímce, které se neprotínají chceme
dokázat, že axiom o přímkách impluje, že nemůžeme žádnou přímku prodloužit tak
aby spojila tyto dva body. 

Kdybychom totiž jakokoliv prodloužili vytvořili bychom tak druhý průsečík mezi
dvěma přímkami. Protože z a1 plyne, že ty dvě přímky jedna na které ležel bod do
kterého jsme prodloužili. a ta druhá kterou jsme prodloužili. už mají jeden
průsečík. My jsme přidali další.

\end{document}
